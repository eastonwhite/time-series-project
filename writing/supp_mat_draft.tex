\documentclass[11pt,]{article}
\usepackage{lmodern}
\usepackage{amssymb,amsmath}
\usepackage{ifxetex,ifluatex}
\usepackage{fixltx2e} % provides \textsubscript
\ifnum 0\ifxetex 1\fi\ifluatex 1\fi=0 % if pdftex
  \usepackage[T1]{fontenc}
  \usepackage[utf8]{inputenc}
\else % if luatex or xelatex
  \ifxetex
    \usepackage{mathspec}
  \else
    \usepackage{fontspec}
  \fi
  \defaultfontfeatures{Ligatures=TeX,Scale=MatchLowercase}
\fi
% use upquote if available, for straight quotes in verbatim environments
\IfFileExists{upquote.sty}{\usepackage{upquote}}{}
% use microtype if available
\IfFileExists{microtype.sty}{%
\usepackage{microtype}
\UseMicrotypeSet[protrusion]{basicmath} % disable protrusion for tt fonts
}{}
\usepackage[margin=0.75in]{geometry}
\usepackage{hyperref}
\hypersetup{unicode=true,
            pdfborder={0 0 0},
            breaklinks=true}
\urlstyle{same}  % don't use monospace font for urls
\usepackage{natbib}
\bibliographystyle{plainnat}
\usepackage{graphicx,grffile}
\makeatletter
\def\maxwidth{\ifdim\Gin@nat@width>\linewidth\linewidth\else\Gin@nat@width\fi}
\def\maxheight{\ifdim\Gin@nat@height>\textheight\textheight\else\Gin@nat@height\fi}
\makeatother
% Scale images if necessary, so that they will not overflow the page
% margins by default, and it is still possible to overwrite the defaults
% using explicit options in \includegraphics[width, height, ...]{}
\setkeys{Gin}{width=\maxwidth,height=\maxheight,keepaspectratio}
\IfFileExists{parskip.sty}{%
\usepackage{parskip}
}{% else
\setlength{\parindent}{0pt}
\setlength{\parskip}{6pt plus 2pt minus 1pt}
}
\setlength{\emergencystretch}{3em}  % prevent overfull lines
\providecommand{\tightlist}{%
  \setlength{\itemsep}{0pt}\setlength{\parskip}{0pt}}
\setcounter{secnumdepth}{5}
% Redefines (sub)paragraphs to behave more like sections
\ifx\paragraph\undefined\else
\let\oldparagraph\paragraph
\renewcommand{\paragraph}[1]{\oldparagraph{#1}\mbox{}}
\fi
\ifx\subparagraph\undefined\else
\let\oldsubparagraph\subparagraph
\renewcommand{\subparagraph}[1]{\oldsubparagraph{#1}\mbox{}}
\fi

%%% Use protect on footnotes to avoid problems with footnotes in titles
\let\rmarkdownfootnote\footnote%
\def\footnote{\protect\rmarkdownfootnote}

%%% Change title format to be more compact
\usepackage{titling}

% Create subtitle command for use in maketitle
\newcommand{\subtitle}[1]{
  \posttitle{
    \begin{center}\large#1\end{center}
    }
}

\setlength{\droptitle}{-2em}
  \title{}
  \pretitle{\vspace{\droptitle}}
  \posttitle{}
  \author{}
  \preauthor{}\postauthor{}
  \date{}
  \predate{}\postdate{}

\usepackage{float} \renewcommand{\thepage}{S\arabic{page}}
\renewcommand{\thesection}{S\arabic{section}}
\renewcommand{\thetable}{S\arabic{table}}
\renewcommand{\thefigure}{S\arabic{figure}}

\begin{document}

\vspace{2cm}

\begin{center}
 \textbf{Supplementary Material: Article title here}
 
Authors: Easton R. White$^{1*}$ , John D. Nagy$^{2,3}$, and Andrew T. Smith$^4$ 
\vspace{3 mm}

Address: \\ \emph{$^1$ Center for Population Biology, University of California-Davis, Davis, California 95616 USA; \\ $^2$ Department of Life Science, Scottsdale Community College, Scottsdale, Arizona 85253 USA; \\ $^3$ School of Mathematical and Statistical Sciences, Arizona State University, Tempe, Arizona 85287 USA; \\ $^4$ School of Life Sciences, Arizona State University, Tempe, Arizona 85287 USA}

*Corresponding author: eawhite@ucdavis.edu

Dec XX, 2016
 \end{center}

\vspace{2cm}

This supplementary material includes:

\begin{enumerate}
\def\labelenumi{\arabic{enumi}.}
\tightlist
\item
  Mathematical description of our model
\item
  Inverse pattern-orientated estimation procedure
\item
  Effects of spatial structure and patch heterogeneity
\item
  Mainland-Island effect
\item
  Invasion from southern area of study site
\item
  Parameter sensitivity
\item
  Effect of unsampled patches
\item
  Long term dynamics
\item
  Individual patch dynamics over time
\end{enumerate}

Code for all the figures and tables can be found on
\href{https://github.com/erwhite1}{Github}.

\vspace{2cm}

\subsection{Example of subsamling and power
calculations}\label{example-of-subsamling-and-power-calculations}

Here, we illustrate in detail how we performed the subsampling and power
calculations for a specific population. As an example, we examine a
35-year time series of (). Simple linear regression indicates a
significant decline for this population with a blank and blank. We
assume that this significant decline over 35 years is in fact the ``true
trend``. In statistical jargon, the 35-year trend is an effect that is
actually present. We can then use this as a benchmark to see if
subsamples of the time series show a similar result.

We first extract all contiguous subsamples of the time series. This
leads to 34 two-year subsamples, 33 two-year subsamples, and so forth
until a single 35-year subsample. We can call each set of subsamples, of
the same length, a set. For each subsample, we conduct linear regression
and extract model coefficients and p-values. Then, the fraction of
subsamples within a set that show significant trends (signfiiant slope
coefficient) is the statistical power. It is important to note that we
only consider subsamples to be significant if they are significant in
the same direction as the complete 35-year time series.

We can then plot statistical power as a function of time series length.
As expected, we can see that power increases with the more years that
are sampled.

Then, we determine an appropriate level of statistical power that we
find acceptable. Traditionally, this has been at 0.8, however, this is
purely historical. Statisical power of 0.8 implies\ldots{}

With statistical power of 0.8, we then determine the minimum time series
length (\(t_min\)) required to achieve that level of statistical power.
Here, \(t_min\) is the first point as which all points to the right are
above 0.8.

\subsection{Results from other questions of
interest}\label{results-from-other-questions-of-interest}

\section{\texorpdfstring{Supplementary material
\ref{fig:growth_rate}}{Supplementary material }}\label{supplementary-material}

\begin{figure}[htbp]
\centering
\includegraphics{supp_mat_draft_files/figure-latex/unnamed-chunk-2-1.pdf}
\caption{duh\label{fig:growth_rate}}
\end{figure}

\bibliography{White_bib.bib}


\end{document}
